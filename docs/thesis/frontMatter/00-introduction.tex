\pagenumbering{roman} % Roman numbers
\chapter*{Introduzione} % "*" means that the chapter is not numbered

% Page heading
\rhead[\fancyplain{}{\bfseries INTRODUZIONE}]{\fancyplain{}{\bfseries\thepage}}
\lhead[\fancyplain{}{\bfseries\thepage}]{\fancyplain{}{\bfseries INTRODUZIONE}}

% Add entry to index
\addcontentsline{toc}{chapter}{Introduzione}

Il monitoraggio della qualità dell'aria urbana rappresenta oggi una delle sfide più rilevanti per la salute
pubblica e la sostenibilità ambientale. In un'epoca caratterizzata dalla crescente urbanizzazione e dall'intensificarsi
delle attività antropiche, la necessità di disporre di sistemi affidabili per il rilevamento e l'interpretazione
dei dati sull'inquinamento atmosferico diventa sempre più necessario. Le concentrazioni di particolato
PM$_{2.5}$ e PM$_{10}$, biossido di azoto, ozono troposferico e anidride solforosa nelle aree metropolitane
influenzano direttamente il benessere dei cittadini, richiedendo strumenti di monitoraggio che siano al contempo
tecnicamente accurati e facilmente accessibili.

L'evoluzione tecnologica degli ultimi anni ha aperto nuove prospettive nell'ambito del monitoraggio ambientale,
rendendo possibile lo sviluppo di sistemi distribuiti di sensori che operano in tempo reale. Parallelamente,
la diffusione delle tecnologie web e dei dispositivi mobili ha creato l'opportunità di democratizzare l'accesso
alle informazioni ambientali, trasformando dati tecnici complessi in visualizzazioni comprensibili e utilizzabili
da un'ampia gamma di utenti. In questo contesto si inserisce il presente elaborato di tesi, che propone lo sviluppo
di AirQualityInsight, un sistema integrato per la simulazione, raccolta e visualizzazione di dati
sulla qualità dell'aria.

Questo applicativo fornisce all'utente una dashboard web che presenta una mappa interattiva di Bologna, sulla quale
vengono disegnate in tempo reale le misurazioni effettuate da un insieme di sensori dislocati sul territorio.
La dashboard web, progettata secondo i principi del design mobile-first, offre un'esperienza utente fluida e intuitiva
sia su dispositivi desktop che mobili. La mappa interattiva, arricchita da layer informativi
sui confini amministrativi e le \acrshort{ztl}, permette agli utenti di esplorare i dati ambientali nel loro contesto
geografico, facilitando la comprensione delle correlazioni spaziali tra inquinamento e caratteristiche territoriali.

Una particolare attenzione è stata dedicata alla visualizzazione dei dati attraverso heatmap dinamiche che traducono
le concentrazioni di inquinanti in rappresentazioni cromatiche intuitive. Questa tecnica di visualizzazione,
basata sulla scala di colori dell'\acrfull{eaqi}, permette di identificare immediatamente le aree critiche e di
monitorare l'evoluzione temporale della qualità dell'aria. L'integrazione di funzionalità di filtraggio e aggregazione
consente agli utenti di personalizzare la visualizzazione in base alle proprie esigenze, concentrandosi
su specifici inquinanti o periodi temporali.

Il presente elaborato si propone di dimostrare come l'integrazione strategica di tecnologie moderne possa dare vita
a sistemi di monitoraggio ambientale che coniughino efficacemente accuratezza tecnica e accessibilità d'uso.
Attraverso l'implementazione di AirQualityInsight, si intende evidenziare le potenzialità delle architetture
distribuite nel contesto delle smart cities, proponendo soluzioni che possano essere replicate e adattate a diverse
realtà urbane. La metodologia adottata, che privilegia l'approccio incrementale e la modularità dei componenti,
rende il sistema facilmente estensibile e manutenibile, caratteristiche fondamentali per la sostenibilità
a lungo termine di infrastrutture tecnologiche destinate al servizio pubblico.

Di seguito viene illustrata la struttura e gli obiettivi di ciascun capitolo del presente elaborato di tesi:

\begin{itemize}
  \item Il \textbf{primo capitolo}~(\ref{chapter:first}) introduce l'obiettivo del progetto di tesi e fornisce
  il quadro teorico necessario alla comprensione del lavoro sviluppato. Viene presentata un'analisi
  dello stato dell'arte dei sistemi esistenti affini alla soluzione proposta, unitamente ai concetti fondamentali
  del dominio applicativo di riferimento. Il capitolo si conclude con una descrizione dettagliata delle funzionalità
  del sistema implementato.
  \item Il \textbf{secondo capitolo}~(\ref{chapter:second}) illustra le tecnologie adoperate
  nello sviluppo del progetto. Considerata la natura web-based dell'applicazione, le tecnologie vengono categorizzate
  distinguendo tra quelle utilizzate per il front-end e quelle impiegate per il back-end.
  \item Il \textbf{terzo capitolo}~(\ref{chapter:third}) presenta in modo approfondito la fase
  di progettazione e implementazione del sistema. Dopo aver delineato l'architettura generale, viene fornita
  una descrizione dettagliata di ciascun componente, dai servizi implementati all'interfaccia utente front-end.
\end{itemize}

\donotmumberlastpageonleft