\clearpage{\pagestyle{empty}\cleardoublepage}
\chapter*{Conclusioni}
\rhead[\fancyplain{}{\bfseries
CONCLUSIONI}]{\fancyplain{}{\bfseries\thepage}}
\lhead[\fancyplain{}{\bfseries\thepage}]{\fancyplain{}{\bfseries
CONCLUSIONI}}

\addcontentsline{toc}{chapter}{Conclusioni}

Il progetto AirQualityInsight ha dimostrato la fattibilità e l'efficacia di un approccio integrato al monitoraggio
ambientale urbano, combinando tecnologie moderne di sviluppo web, architetture distribuite e tecniche avanzate
di visualizzazione dati. L'implementazione del sistema ha permesso di validare le scelte architetturali adottate e di
confermare l'importanza della user experience nella progettazione di strumenti per il monitoraggio ambientale
destinati al pubblico generale.

Uno degli aspetti più significativi emersi durante lo sviluppo è stata l'importanza della geolocalizzazione
strategica dei sensori. L'utilizzo delle query Overpass per identificare le intersezioni stradali più rilevanti e la
successiva applicazione di algoritmi di filtraggio spaziale hanno permesso di ottimizzare la copertura territoriale,
bilanciando efficacemente densità di monitoraggio e utilizzo delle risorse. La scelta di concentrarsi sui punti
di maggiore criticità del traffico urbano si è dimostrata una strategia interessante per massimizzare l'utilità
informativa del sistema.

L'implementazione delle heatmap dinamiche ha evidenziato le potenzialità delle visualizzazioni geografiche nel rendere
accessibili informazioni tecniche complesse. La capacità di tradurre concentrazioni di inquinanti in rappresentazioni
cromatiche intuitive ha trasformato dataset numerici in strumenti di comunicazione efficaci e rappresentativi,
facilitando la comprensione immediata delle condizioni ambientali e dei loro pattern spaziali. L'integrazione con gli
standard europei \acrshort{eaqi} ha inoltre garantito la coerenza con i sistemi di monitoraggio ufficiali,
aumentando la credibilità e l'utilità pratica del sistema.

Il progetto ha tuttavia evidenziato anche alcune limitazioni e aree di possibile miglioramento. La natura simulata
dei dati, pur permettendo lo sviluppo e il testing del sistema, non può sostituire completamente l'integrazione
con sensori reali per una validazione completa dell'efficacia operativa. Inoltre, l'attuale implementazione
si concentra esclusivamente sul territorio bolognese, limitando la valutazione della scalabilità geografica del sistema.

Le fondamenta architetturali e tecnologiche di AirQualityInsight aprono numerose prospettive per
estensioni e miglioramenti futuri, ciascuna delle quali potrebbe significativamente ampliare
l'utilità e l'impatto del sistema.
Di seguito si propongo degli sviluppi futuri che potrebbero arricchire il sistema realizzato in modo da accrescerne
il potere informativo e migliorare l'esperienza utente con l'aggiunta di nuove funzionalità.

Le evoluzioni principali riguardano l'implementazione di funzionalità di analisi retrospettiva con controlli temporali
granulari per visualizzare l'evoluzione della qualità dell'aria su diverse scale temporali, dall'analisi oraria a
quella mensile. Questa funzionalità richiederebbe algoritmi di compressione e aggregazione dei dati storici
per garantire prestazioni ottimali.

Un sistema di notifiche proattivo trasformerebbe il sistema da strumento passivo a piattaforma di allerta attiva,
implementando soglie personalizzabili e meccanismi di notifica multicanale. Gli algoritmi predittivi potrebbero
anticipare il superamento delle soglie critiche, integrando servizi di messaggistica esterni
per garantire l'affidabilità delle comunicazioni.

Il modulo di reporting avanzato beneficerebbe enti pubblici e ricercatori attraverso la generazione automatica
di grafici statistici e report personalizzabili, trasformando i dati grezzi in strumenti di supporto decisionale.
L'implementazione includerebbe librerie di visualizzazione avanzate e template conformi agli standard normativi europei.

I servizi di prossimità geografica fornirebbero informazioni sulla qualità dell'aria nel raggio
di distanze personalizzabili, utilizzando \acrshort{api} di geolocalizzazione e algoritmi di interpolazione spaziale
per calcolare indici medi in tempo reale, risultando particolarmente utili per applicazioni mobile.

L'evoluzione più ambiziosa riguarda lo sviluppo di capacità predittive tramite modelli di machine learning
che integrino variabili meteorologiche, pattern di traffico e dati storici per fornire
previsioni affidabili a breve e medio termine.

L'integrazione con l'ecosistema urbano includerebbe l'interoperabilità con sistemi smart city
esistenti attraverso \acrshort{api} standardizzate, mentre l'espansione geografica oltre Bologna richiederebbe
meccanismi automatizzati per l'acquisizione di dati via \acrlong{osm}. Infine, l'evoluzione verso un monitoraggio
collaborativo integrerebbe sensori citizen science e dispositivi \acrshort{iot} domestici, democratizzando l'accesso
alle informazioni ambientali.

La modularità del sistema e l'adozione di standard aperti facilitano un'implementazione
incrementale che bilanci innovazione tecnologica e sostenibilità operativa.